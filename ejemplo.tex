\documentclass[letterpaper,12pt]{article}

\usepackage[utf8]{inputenc} % Soporte para acentos
\usepackage[T1]{fontenc}    
\usepackage[spanish]{babel} % Español
\usepackage{amsmath}	% Soporte de símbolos adicionales (matemáticas)
\usepackage{amssymb}		
\usepackage{amsfonts}
\usepackage{latexsym}

% Información para el título
\title{Documento modular}
\author{Miguel Piña}

% Indicamos una separación entre los párrafos
\parskip=3mm

% Eliminamos la sangría de los párrafos
\parindent=0mm

\pagestyle{myheadings}
%\markright{\LaTeX \hfill Fórmulas matemáticas \;\;}

\begin{document}

\maketitle
% con \input mandamos llamar docuemntos en insertarlos y se insertan en el orden en el que los coloquemos si escribimos \include nos da un salto de línea
\include{miprimeraseccion/etimologia}	% Llama al documetno etimologia y lo inserta
\section{Crédito revolvente}

Los clientes de tarjetas de crédito pueden tener diferentes formas para pagar el uso de su línea de crédito.
Por lo general será en cuotas o en modalidad revolving. Los clientes que tienen modalidad revolving pueden 
realizar un pago menor al total facturado en el período (llamado Pago Mínimo). El saldo (la diferencia entre 
lo facturado y lo pagado), genera una nueva deuda (revolving) a la que se le aplica la tasa de interés vigente 
para el período y se adiciona al saldo de deuda de esta modalidad, correspondientes a los períodos anteriores 
si existieren. Esta deuda puede ser pagada (amortizada) por el cliente de manera diferida en el tiempo.\cite{banco1991informe}
		% Llama al documento section y lo inserta
\section{Tipos de créditos}

\begin{itemize}
	\item Crédito tradicional: Préstamo que contempla un pie y un número de cuotas a convenir. Habitualmente estas cuotas incluyen seguros ante cualquier siniestro involuntario.
	\item Crédito al consumo: Préstamo a corto o mediano plazo (1 a 4 años) que sirve para adquirir bienes o cubrir pago de servicios.
	\item Crédito comercial: Préstamo que se realiza a empresas de indistinto tamaño para la adquisición de bienes, pago de servicios de la empresa o para refinanciar deudas con otras instituciones y proveedores de corto plazo.
	\item Crédito hipotecario: Dinero que entrega el banco o financiera para adquirir una propiedad ya construida, un terreno, la construcción de viviendas, oficinas y otros bienes raíces, con la garantía de la hipoteca sobre el bien adquirido o construido; normalmente es pactado para ser pagado en el mediano o largo plazo (8 a 40 años, aunque lo habitual son 20 años).
	\item Crédito consolidado: Es un préstamo que reúne todos los otros préstamos que un prestatario tiene en curso, en un único y nuevo crédito. Habitualmente estos préstamos consolidados permiten a quienes los suscriben pagar una cuota periódica inferior a la suma de las cuotas de los préstamos separados, si bien en contraprestación suele prolongarse el plazo del crédito y/o el tipo de interés a aplicar.
	\item Crédito personal: Dinero que entrega el banco o financiera a un individuo , persona física, y no a personas jurídica,para adquirir un bien mueble (entiéndase así por bienes que no sean propiedades/viviendas), el cual puede ser pagado en el mediano o corto plazo (1 a 6 años).
	\item Crédito prendario: Dinero que le entrega el banco o entidad financiera a una persona física, y no a personas jurídicas para efectuar la compra de un bien mueble, generalmente el elemento debe de ser aprobado por el banco o entidad financiera, y puesto que este bien mueble a comprar quedara con una prenda, hasta una vez saldada la deuda con la entidad financiera o Bancaria.
	\item Crédito rápido: Es un tipo de préstamo que suelen comercializar entidades financieras de capital privado, de baja cuantía y cierta flexibilidad en los plazos de amortización, convirtiéndose en productos atractivos sobre todo en casos de necesidades urgentes de liquidez.
	\item Mini Crédito: Préstamo de baja cuantía (hasta 600 euros) a devolver en no más de 30 días que conceden las entidades de crédito. Se caracterizan por su solicitud ágil, su aprobación o denegación rápidas y por ser bastante más caros que los préstamos bancarios. 
\end{itemize}
		% Llama al documento llamado credito y lo inserta
\bibliographystyle{plain}
\bibliography{biblio}
\end{document}
