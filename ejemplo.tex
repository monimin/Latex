\documentclass[letterpaper,12pt]{article}

\usepackage[utf8]{inputenc} % Soporte para acentos
\usepackage[T1]{fontenc}    
\usepackage[spanish]{babel} % Español
\usepackage{amsmath}	% Soporte de símbolos adicionales (matemáticas)
\usepackage{amssymb}		
\usepackage{amsfonts}
\usepackage{latexsym}

% Información para el título
\title{Documento modular}
\author{Miguel Piña}

% Indicamos una separación entre los párrafos
\parskip=3mm

% Eliminamos la sangría de los párrafos
\parindent=0mm

\pagestyle{myheadings}
%\markright{\LaTeX \hfill Fórmulas matemáticas \;\;}

\begin{document}

\maketitle
% con \input mandamos llamar docuemntos en insertarlos y se insertan en el orden en el que los coloquemos si escribimos \include nos da un salto de línea
\include{miprimeraseccion/etimologia}	% Llama al documetno etimologia y lo inserta
\include{section}		% Llama al documento section y lo inserta
\include{credito}		% Llama al documento llamado credito y lo inserta
\bibliographystyle{plain}
\bibliography{biblio}
\end{document}
